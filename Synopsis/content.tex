
\section*{Общая характеристика работы}

\newcommand{\actuality}{\underline{\textbf{\actualityTXT}}}
\newcommand{\progress}{\underline{\textbf{\progressTXT}}}
\newcommand{\aim}{\underline{{\textbf\aimTXT}}}
\newcommand{\tasks}{\underline{\textbf{\tasksTXT}}}
\newcommand{\novelty}{\underline{\textbf{\noveltyTXT}}}
\newcommand{\influence}{\underline{\textbf{\influenceTXT}}}
\newcommand{\methods}{\underline{\textbf{\methodsTXT}}}
\newcommand{\defpositions}{\underline{\textbf{\defpositionsTXT}}}
\newcommand{\reliability}{\underline{\textbf{\reliabilityTXT}}}
\newcommand{\probation}{\underline{\textbf{\probationTXT}}}
\newcommand{\contribution}{\underline{\textbf{\contributionTXT}}}
\newcommand{\publications}{\underline{\textbf{\publicationsTXT}}}


{\actuality} В настоящий время каждый поцтреотически настроенный гражданин любой славянской страны готов часами и с пеной у рта доказывать, что рассматриваемый в данной работе субстрат изобрели именно у него на родине. Изредка, если оппонент настолько туп, что не понимает его (поцреотову) правоту, а он (поцреот) внезапно добр, он готов признать, что в других славянских странах тоже есть свой национальный субстрат, но только у него в стране сохранилась исконная рецептура оного, и вообще он вкуснее.
Что же касается стран, не имеющих отношения к СНГ, то они гораздо спокойнее относятся к теме вайна по поводу авторских прав на оный субстрат, втайне зная, что сей субстрат таки изобретён у них.
% {\progress} 
% Этот раздел должен быть отдельным структурным элементом по
% ГОСТ, но он, как правило, включается в описание актуальности
% темы. Нужен он отдельным структурынм элемементом или нет ---
% смотрите другие диссертации вашего совета, скорее всего не нужен.

{\aim} разработка технологического процесса изготовления неоднородного субстрата из растительных и животных компонентов, который должен удовлетворять необходими условиям для троллинга в пределах СНГ.

Для достижения поставленной цели необходимо было решить следующие {\tasks}:

\begin{enumerate}
  \item Исследовать современные условия троллинга субстратом из растительных и животных компонентов в переделах СНГ.
  \item Выделить ключевые компоненты, из которых данный субстрат готовится.
  \item Разработать технологический процесс приготовления субстрата, удовлетворяющего выделенным условиям троллинга и вклчающий в себе все необходимые компоненты.
  \item Провести испытание полученного с помощью разработанного технологического процесса субстрата и оценить качество троллинга.
\end{enumerate}


{\novelty}
\begin{enumerate}
  \item Впервые \ldots
  \item Впервые \ldots
  \item Было выполнено оригинальное исследование \ldots
\end{enumerate}

{\influence} \ldots

{\methods} \ldots

{\defpositions}
\begin{enumerate}
  \item Первое положение
  \item Второе положение
  \item Третье положение
  \item Четвертое положение
\end{enumerate}
В папке Documents можно ознакомиться в решением совета из Томского ГУ
в файле \verb+Def_positions.pdf+, где обоснованно даются рекомендации
по формулировкам защищаемых положений. 

{\reliability} полученных результатов обеспечивается \ldots \ Результаты находятся в соответствии с результатами, полученными другими авторами.


{\probation}
Основные результаты работы докладывались~на:
перечисление основных конференций, симпозиумов и~т.\:п.

{\contribution} Автор принимал активное участие \ldots

%\publications\ Основные результаты по теме диссертации изложены в ХХ печатных изданиях~\cite{Sokolov,Gaidaenko,Lermontov,Management},
%Х из которых изданы в журналах, рекомендованных ВАК~\cite{Sokolov,Gaidaenko}, 
%ХХ --- в тезисах докладов~\cite{Lermontov,Management}.

\ifnumequal{\value{bibliosel}}{0}{% Встроенная реализация с загрузкой файла через движок bibtex8
    \publications\ Основные результаты по теме диссертации изложены в XX печатных изданиях, 
    X из которых изданы в журналах, рекомендованных ВАК, 
    X "--- в тезисах докладов.%
}{% Реализация пакетом biblatex через движок biber
%Сделана отдельная секция, чтобы не отображались в списке цитированных материалов
    \begin{refsection}[vak,papers,conf]% Подсчет и нумерация авторских работ. Засчитываются только те, которые были прописаны внутри \nocite{}.
        %Чтобы сменить порядок разделов в сгрупированном списке литературы необходимо перетасовать следующие три строчки, а также команды в разделе \newcommand*{\insertbiblioauthorgrouped} в файле biblio/biblatex.tex
        \printbibliography[heading=countauthorvak, env=countauthorvak, keyword=biblioauthorvak, section=1]%
        \printbibliography[heading=countauthorconf, env=countauthorconf, keyword=biblioauthorconf, section=1]%
        \printbibliography[heading=countauthornotvak, env=countauthornotvak, keyword=biblioauthornotvak, section=1]%
        \printbibliography[heading=countauthor, env=countauthor, keyword=biblioauthor, section=1]%
        \nocite{%Порядок перечисления в этом блоке определяет порядок вывода в списке публикаций автора
                vakbib1,vakbib2,%
                confbib1,confbib2,%
                bib1,bib2,%
        }%
        \publications\ Основные результаты по теме диссертации изложены в \arabic{citeauthor} печатных изданиях, 
        \arabic{citeauthorvak} из которых изданы в журналах, рекомендованных ВАК, 
        \arabic{citeauthorconf} "--- в тезисах докладов.
    \end{refsection}
    \begin{refsection}[vak,papers,conf]%Блок, позволяющий отобрать из всех работ автора наиболее значимые, и только их вывести в автореферате, но считать в блоке выше общее число работ
        \printbibliography[heading=countauthorvak, env=countauthorvak, keyword=biblioauthorvak, section=2]%
        \printbibliography[heading=countauthornotvak, env=countauthornotvak, keyword=biblioauthornotvak, section=2]%
        \printbibliography[heading=countauthorconf, env=countauthorconf, keyword=biblioauthorconf, section=2]%
        \printbibliography[heading=countauthor, env=countauthor, keyword=biblioauthor, section=2]%
        \nocite{vakbib2}%vak
        \nocite{bib1}%notvak
        \nocite{confbib1}%conf
    \end{refsection}
}
При использовании пакета \verb!biblatex! для автоматического подсчёта
количества публикаций автора по теме диссертации, необходимо
их здесь перечислить с использованием команды \verb!\nocite!.
    

 % Характеристика работы по структуре во введении и в автореферате не отличается (ГОСТ Р 7.0.11, пункты 5.3.1 и 9.2.1), потому её загружаем из одного и того же внешнего файла, предварительно задав форму выделения некоторым параметрам

%Диссертационная работа была выполнена при поддержке грантов ...

%\underline{\textbf{Объем и структура работы.}} Диссертация состоит из~введения, четырех глав, заключения и~приложения. Полный объем диссертации \textbf{ХХХ}~страниц текста с~\textbf{ХХ}~рисунками и~5~таблицами. Список литературы содержит \textbf{ХХX}~наименование.

%\newpage
\section*{Содержание работы}
Во \underline{\textbf{введении}} обосновывается актуальность
исследований, проводимых в рамках данной диссертационной работы,
приводится обзор научной литературы по изучаемой проблеме,
формулируется цель, ставятся задачи работы, излагается научная новизна
и практическая значимость представляемой работы. В последующих главах
сначала описывается общие принципы получения борща для троллинга, оисывается разработанный технологический процесс производства борща и вывод его общей формулы, а затем идёт апробация на частных примерах троллинга полученным субстратом на примере стран СНГ.

\underline{\textbf{Первая глава}} посвящена историческому обзору процесса изготовления субстрата из растительных и животных компонентов на примере борща в Российской империи в период с начала XVIII по конец XIX века и СССР. А так же рассматривается троллинг борщем за переделами СССР. На рисунке 1 представлен самый известный пример троллинга борщем.

\begin{figure}[ht] 
  \center
  \includegraphics [scale=0.27] {stalin_borsht}
  \caption{Троллинг борщем в США по случаю смерти И.В. Сталина в марте 1953 года} 
  \label{img:stalin_borsht}
\end{figure}

Формулы в строку без номера добавляются так:
\[ 
  \lambda_{T_s} = K_x\frac{d{x}}{d{T_s}}, \qquad
  \lambda_{q_s} = K_x\frac{d{x}}{d{q_s}},
\]

\underline{\textbf{Вторая глава}} посвящена исследованию 

\underline{\textbf{Третья глава}} посвящена исследованию 

В \underline{\textbf{четвертой главе}} приведено описание 

В \underline{\textbf{заключении}} приведены основные результаты работы, которые заключаются в следующем:
\input{common/concl}


%\newpage
%При использовании пакета \verb!biblatex! список публикаций автора по теме
%диссертации формируется в разделе <<\publications>>\ файла
%\verb!../common/characteristic.tex!  при помощи команды \verb!\nocite! 

\ifdefmacro{\microtypesetup}{\microtypesetup{protrusion=false}}{} % не рекомендуется применять пакет микротипографики к автоматически генерируемому списку литературы
\ifnumequal{\value{bibliosel}}{0}{% Встроенная реализация с загрузкой файла через движок bibtex8
  \renewcommand{\bibname}{\large \authorbibtitle}
  \nocite{bib1} Идентификация взаимосвязей между терминами и объектами экономической тематики в тексте на естественном языке / А.С. Дмитриев, И.С. Соловьев, Ю.А. Орлова, В.Л. Розалиев, В.М. Константинов // Прикаспийский журнал: управление и высокие технологии. - 2015. - № 4. - C. 198-209.
  \insertbiblioauthor           % Подключаем Bib-базы
  %\insertbiblioother   % !!! bibtex не умеет работать с несколькими библиографиями !!!
}{% Реализация пакетом biblatex через движок biber
  \insertbiblioauthor           % Вывод всех работ автора
%  \insertbiblioauthorgrouped    % Вывод всех работ автора, сгруппированных по источникам
%  \insertbiblioauthorimportant  % Вывод наиболее значимых работ автора (определяется в файле characteristic во второй section)
  \insertbiblioother            % Вывод списка литературы, на которую ссылались в тексте автореферата
}
\ifdefmacro{\microtypesetup}{\microtypesetup{protrusion=true}}{}

{\textbf{Список опубликованных работ по теме диссертации.}}

{\textbf{1.}}  Идентификация взаимосвязей между терминами и объектами экономической тематики в тексте на естественном языке/ А.С. Дмитриев, И.С. Соловьев, Ю.А. Орлова, В.Л. Розалиев, В.М. Константинов // Прикаспийский журнал: управление и высокие технологии. - 2015. - № 4. - C. 198-209.

{\textbf{2.}} Development of 3D Human Body Model / В.М. Константинов, В.Л. Розалиев, Ю.А. Орлова, А.В. Заболеева-Зотова // Proceedings of the First International Scientific Conference «Intelligent Information Technologies for Industry» (IITI’16) (Rostov-on-Don – Sochi, Russia, May 16-21, 2016). Vol. 2 / ed. by A. Abraham [etc.]. – [Switzerland] : Springer International Publishing, 2016. – P. 143-152. – (Ser. Advances in Intelligent Systems and Computing ; Vol. 451).

{\textbf{3.}} Константинов, В.М. Автоматизированная система создания рукописных документов / В.М. Константинов, В.Л. Розалиев, И.А. Дианов // Открытые семантические технологии проектирования интеллектуальных систем = Open Semantic Technologies for Intelligent Systems (OSTIS-2013) : матер. III междунар. науч.-техн. конф. (Минск, 21-23 февр. 2013 г.) / УО "Белорусский гос. ун-т информатики и радиоэлектроники", ГУ "Администрация Парка высоких технологий". – Минск, 2013. – С. 357-360.

{\textbf{4.}} Константинов, В.М. Разработка 3D модели тела человека с использованием MS Kinect / В.М. Константинов, А.В. Заболеева-Зотова // России – творческую молодёжь : тез. докл. VIII регион. науч.-практ. студенч. конф., посвящ. 70-летию Победы в Великой Отечественной войне (г. Камышин, 22-23 апр. 2015 г.). В 2 т. Т. 1 / ВолгГТУ, КТИ (филиал) ВолгГТУ. - Волгоград, 2015. - C. 187.

{\textbf{5.}} Константинов, В.М. Разработка 3D-модели тела человека с использованием MS Kinect / В.М. Константинов, Ю.А. Орлова, В.Л. Розалиев // Известия ВолгГТУ. Сер. Актуальные проблемы управления, вычислительной техники и информатики в технических системах. - Волгоград, 2015. - № 6 (163). - C. 65-69.

{\textbf{6.}} Константинов, В.М. Разработка 3D-модели тела человека с использованием MSKinect / В.М. Константинов // Современные технологии и управление : сб. науч. тр. III междунар. науч.-практ. конф. (20-21 нояб. 2014 г.) / ФГБОУ ВО Московский гос. ун-т технологий и управления им. К.Г. Разумовского (Первый казачий ун-т), Филиал в р. п. Светлый Яр Волгоградской области. - Светлый Яр, 2014. - C. 108-111.

{\textbf{7.}} Константинов, В.М. Синтез рукописного текста / В.М. Константинов, В.Л. Розалиев, И.А. Дианов // Интегрированные модели и мягкие вычисления в искусственном интеллекте : сб. науч. тр. VII-й междунар. науч.-практ. конф. (Коломна, 20-22 мая 2013 г.). В 3 т. Т. 3 / Рос. ассоциация искусств. интеллекта, Рос. ассоциация нечётких систем и мягких вычислений, МГТУ им. Н.Э. Баумана [и др.]. - М., 2013. - C. 1257-1264.

{\textbf{8.}} Рогудеев, А.Б. Анализ моделей механизмов в Autodesk Inventor на избыточные связи / А.Б. Рогудеев, В.В. Сомов, В.М. Константинов // Прогресс транспортных средств и систем – 2013 : матер. междунар. науч.-практ. конф., Волгоград, 24-26 сент. 2013 г. / ВолгГТУ [и др.]. - Волгоград, 2013. - C. 335.

{\textbf{9.}} Выделение и подсчёт избыточных связей при построении моделей в Autodesk Inventor / А.Б. Рогудеев, В.В. Сомов, В.М. Константинов, А.О. Пивоваров, А.В. Малолетов // Известия ВолгГТУ. Серия "Актуальные проблемы управления, вычислительной техники и информатики в технических системах". Вып. 19 : межвуз. сб. науч. тр. / ВолгГТУ. - Волгоград, 2013. - № 24 (127). - C. 74-79.
